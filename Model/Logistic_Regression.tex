\documentclass[14pt,a4paper]{article}
\usepackage[a4paper, total={6in, 8in}]{geometry}
\usepackage[utf8]{inputenc}
\usepackage{amsmath}
\usepackage{amsfonts}
\usepackage{amssymb}
\usepackage{graphicx,mathtools}
\usepackage{enumerate} 
\usepackage{fixmath} 


\newcommand{\equparrow}[1][1]{\stackrel{\text{\scalebox{1}[#1]{$\uparrow$}}}{=}}
\newcommand{\equparrowx}[2][1]{\stackrel{\mathclap{#2}}{\equparrow[#1]}}
\author{Sara Zapata-Marin}
\title{Logistic Regression model to estimate voxel activation}
\begin{document}

\maketitle

We let $Y_{ij}$ be the $j-$th voxel of study $i$,

\[ y_{ij}=
\begin{cases}
      1 \qquad  \text{if is \textit{on} in the \textit{i}-\text{th} study}\\
      0 \qquad  \text{if is \textit{off} in the \textit{i}-\text{th} study}
    \end{cases} \]
For $i=1,2,....,I$.
With one voxel we can assume that it follows a Bernoulli distribution,

\[
Y_{ij} \sim Bernoulli(\pi_{ij})
\]

\[ log\dfrac{\pi_{ij}}{1-\pi_{ij}}= \beta_0+\sum_{k=1}^p \beta_k X_{ijk},\]

where $X=(X_{ij1},...,X_{ijp})$ are possible covariates that influence the activation of the voxel $j$. Information about the possible covariates of each study will be needed.
\\ After estimation $\pi_{ij}$ we can build our prior based on the average values of the $\pi_{ij}$'s over studies. Our probability surface would be estimated as
\[ \pi_j=\dfrac{1}{I}\sum_{i=1}^I \pi_{ij}\]
\\We need to assign the prior distribution of $\beta=(\beta_0,...,\beta_p)$. It is reasonable to assume that the $\beta_k$ are independend and follow a zero mean normal distribution with large and fixed variance. 
\\For the posterior distribution we will use a Markov Chain Monte Carlo method to obtain samples from the resultant posterior distribution.
\end{document}